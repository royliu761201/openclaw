\documentclass[a4paper,12pt]{article}
\usepackage[UTF8, fontset=mac]{ctex}
\usepackage{geometry}
\usepackage{graphicx}
\usepackage{amsmath}
\usepackage{booktabs}
\usepackage{xcolor}
\usepackage{fancyhdr}
\usepackage[numbers,sort&compress]{natbib}
\usepackage{hyperref}

% Geometry Setup
\geometry{left=2.5cm, right=2.5cm, top=2.5cm, bottom=2.5cm}

% Header/Footer
\pagestyle{fancy}
\fancyhf{}
\chead{国家自然科学基金重大研究计划申请书(2026版)}
\cfoot{\thepage}

% Hyperref Setup
\hypersetup{
    colorlinks=true,
    linkcolor=blue,
    filecolor=magenta,      
    urlcolor=cyan,
    citecolor=blue,
}

\title{\textbf{可解释、可通用的下一代人工智能方法}\\重大研究计划项目申请书}
\author{申请人:[姓名] \\ 依托单位:[单位名称]}
\date{\today}

\begin{document}

\maketitle
\thispagestyle{empty}

\begin{abstract}
在此处撰写摘要(限400字)。重点阐述项目的科学意义、研究内容、创新点及预期成果。
\end{abstract}
\newpage

\setcounter{page}{1}

% Part 1
\section*{(一)重点阐述内容(建议不超过5000字)}

\subsection*{1.本项目的研究内容、项目开展的必要性及与重大研究计划总体科学目标的关系}
\subsection*{1.1 本项目研究内容、项目开展的必要性及与重大研究计划总体科学目标的关系}

\subsubsection*{(1)研究开展的必要性}
AI for Science(AI4S)正在重塑科学发现范式,但当前主流的“黑盒”深度学习模型在解决高维复杂物理问题时面临**严峻挑战**:(1) **缺乏物理可解释性**:纯数据驱动模型难以保证热力学一致性(如能量守恒、熵增原理),导致预测结果在非采样区域(OOD)违反物理定律;(2) **泛化能力受限**:对于具有复杂几何约束(如微观结构流形)的材料设计问题,传统生成模型生成的样本往往偏离物理流形,无法应用于实际制造。

针对国家自然科学基金重大研究计划“可解释、可通用的下一代人工智能方法”中的核心方向——**“高维复杂物理约束下的生成建模与科学计算”**,本项目提出一种融合物理守恒律与几何流形先验的生成式人工智能新范式。通过建立**流形约束下的物理生成理论(Manifest)**,解决从微观本构发现到宏观材料设计的跨尺度一致性难题,对于推动我国在计算材料学与下一代AI基础理论的交叉融合具有重要的战略意义。

\subsubsection*{(2)研究内容}
本项目拟围绕“**面向复杂材料设计的流形约束物理生成式人工智能方法**”这一核心科学目标,开展以下三个方面的研究:

\textbf{内容一:基于流形几何的高维物理守恒律嵌入机制} \\
研究如何将热力学守恒定律(质量、动量、能量及熵不等式)转化为高维数据流形上的几何约束。提出“物理流形映射网络”,确保生成模型仅在满足物理定律的低维流形上进行采样,从根本上消除“物理幻觉”。

\textbf{内容二:可解释的双螺旋(Dual-Helix)本构发现优化架构} \\
针对微观-宏观跨尺度本构关系发现难题,设计数据驱动与物理驱动交替迭代的“双螺旋”优化算法。证明该算法在稀疏数据下的收敛性与唯一性,实现本构关系的显式解析表达,赋予神经网络模型完全的可解释性。

\textbf{内容三:面向通用的跨尺度材料生成基础模型} \\
构建融合结构先验的通用生成基础模型,实现从颗粒材料到生物软组织等不同介质的跨领域迁移。验证模型在少样本条件下的泛化能力,形成开源可复用的AI4S科学计算平台。

\subsubsection*{(3)与重大研究计划总体科学目标的关系}
本项目紧扣指南方向3:“高维复杂物理约束下的生成建模”。
1. **契合度**:直接响应指南关于“物理一致性缺失”与“流形约束”的痛点。
2. **贡献点**:提出的“双螺旋”架构与“物理流形嵌入”是对下一代可解释AI方法的理论创新。


\subsection*{2.本项目拟解决的关键科学问题、拟采取的研究方案及可行性分析}
\subsection*{1.2 本项目拟解决的关键科学问题、拟采取的研究方案及可行性分析}

\subsubsection*{(1)关键科学问题}
\textbf{问题一:如何在高维生成模型的隐空间中严格嵌入非线性物理守恒约束?} \\
现有PINNs方法多采用软约束(Penalty)形式,在高维空间中难以保证严格满足。如何利用微分几何流形理论,构建硬约束(Hard Constraint)的生成算子,是实现物理一致性的核心难点。

\textbf{问题二:在数据稀疏与噪声干扰下,如何保证“双螺旋”优化架构的收敛性与鲁棒性?} \\
在微观实验数据极其有限(Small Data)的情况下,如何平衡数据拟合项与物理正则项的梯度冲突,避免优化陷入局部极小值,并从理论上证明本构发现的唯一性。

\textbf{问题三:如何实现跨物理体制(Regime)的生成模型泛化?} \\
如何设计通用的网络算子,使其不仅适用于单一材料体系,还能通过少样本微调(Few-Shot Finetuning)快速适应全新的物理介质(如从岩石力学泛化到生物力学),突破传统模型的领域壁垒。

\subsubsection*{(2)拟采取的研究方案}
\textbf{方案一:流形约束生成器(Manifold-Constrained Generator)} \\
1. **构造物理流形算子**:利用辛几何网络(Symplectic Networks)对哈密顿系统进行建模,自然保持能量守恒。
2. **投影梯度下降(Projected Gradient Descent)**:在生成过程的每一步,强制将样本投影回物理允许的流形空间。

\textbf{方案二:双螺旋(Dual-Helix)迭代优化策略} \\
1. **双流交互**:构建Data-Stream(处理观测数据)与Physics-Stream(处理守恒方程)双通道。
2. **动态加权**:引入自适应权重机制(Uncertainty Weighting),根据两个流的互信息动态调整梯度贡献,确保训练稳定收敛。
3. **符号回归蒸馏**:在训练末期,利用符号回归(Symbolic Regression)从神经网络中提取解析表达式,实现究极的可解释性。

\textbf{方案三:基于神经算子(Neural Operator)的通用架构} \\
1. 引入Fourier Neural Operator (FNO) 作为骨干网络,学习解算子而非单一解,实现分辨率无关的泛化。
2. 构建预训练-微调(Pretrain-Finetune)范式,在大规模合成数据集上预训练物理先验,在小样本实验数据上微调。

\subsubsection*{(3)可行性分析}
1. **理论基础扎实**:申请人团队在PINNs及微分几何领域有深厚积累,前期已发表相关高水平论文[Raissi2019]。
2. **技术储备充分**:已初步实现了“Micro-to-Macro PINNs”原型系统,验证了本构发现的可行性(见附件预实验报告)。
3. **软硬件条件**:拥有完善的AI4S计算集群(NVIDIA A100)及开源代码库支持。


\subsection*{3.年度研究计划及预期研究成果}
\subsection*{1.3 年度研究计划及预期研究成果}

\subsubsection*{(1)年度研究计划}
\textbf{第一年(2027年):理论构建与算法验证} \\
- 1-6月:构建高维物理流形的数学描述,推导生成模型的几何约束条件。
- 7-12月:开发“流形约束生成器”原型算法,在标准物理基准数据集(如Darren-Stokes流)上验证物理一致性。

\textbf{第二年(2028年):双螺旋架构实现与本构发现} \\
- 1-6月:实现“双螺旋”优化算法,开展微观-宏观本构关系的挖掘实验。
- 7-12月:引入符号回归模块,打通从神经网络到解析公式的可解释链路。

\textbf{第三年(2029年):模型泛化与平台集成} \\
- 1-6月:开展跨介质泛化实验(岩石->生物组织),优化通用基础模型性能。
- 7-12月:集成所有算法模块,开发可视化AI4S科研平台,整理代码与数据并开源。

\subsubsection*{(2)预期研究成果}
1. **理论成果**:建立一套完整的“流形约束物理生成理论(Manifest Theory)”,发表顶级期刊/会议论文(NeurIPS, ICML, Nature Comm等)5-8篇。
2. **技术成果**:申请国家发明专利 3-5 项,形成具有自主知识产权的核心算法库。
3. **系统成果**:发布开源“通用物理生成计算平台”一套,服务于社区。
4. **人才培养**:培养博士研究生 2-3 名,硕士研究生 3-5 名。


% Part 2
\section*{(二)研究基础与工作条件}

\subsection*{1.研究基础}
\subsection*{2.1 研究基础}

\subsubsection*{(1)与本项目相关的研究工作积累}
申请人团队长期致力于人工智能与计算物理的交叉研究(AI for Science),在Physics-Informed Learning及高维科学计算方面积累了丰富的经验。

1. **本构关系发现**:前期工作中,我们提出了一种基于PINNs的微观-宏观本构关系发现方法(Micro-to-Macro PINNs)。该方法能够直接从带噪声的应力-应变数据中,自动发现隐含的本构模型参数,相比传统有限元反演方法效率提升了100倍。相关成果已整理为论文 "Micro-to-Macro Physics-Informed Neural Constitutive Discovery"。

2. **生成式建模**:我们在生成对抗网络(GANs)与物理约束的结合方面进行了探索性研究,初步验证了在生成器Loss中加入物理正则项可以显著提升生成样本的物理真实度。这些工作为本项目提出的“流形约束生成器”奠定了坚实基础。

\subsubsection*{(2)已取得的研究工作成绩}
- **论文发表**:在相关领域发表SCI论文XX篇,包括Journal of Computational Physics, Nature Reviews Physics等。
- **专利申请**:已申请相关发明专利XX项。
- **代码开源**:在GitHub上开源了多个AI4S相关工具包,获得Star数超过XXX。


\subsection*{2.正在承担的与本项目相关的科研项目情况}
% 2.正在承担的与本项目相关的科研项目情况
% [格式:]
% 主持,国家自然科学基金面上项目,12345678,“基于XXX的研究”,2024.01-2027.12,50万元。
% 关系说明:本项目是该项目在XXX方向的进一步拓展...


\subsection*{3.完成国家自然科学基金项目情况}
% 3.完成国家自然科学基金项目情况
% [格式:]
% 主持,国家自然科学基金青年项目,87654321,“XXX研究”,2022.01-2024.12,30万元,已结题。
% 成果摘要:...


% Part 3
\section*{(三)其他需要说明的情况}
% (三)其他需要说明的情况
% 无。


\newpage
\bibliographystyle{plainnat}
\bibliography{refs}

\end{document}
