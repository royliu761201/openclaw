\subsection*{2.1 研究基础}

\subsubsection*{(1)与本项目相关的研究工作积累}
申请人团队长期致力于人工智能与计算物理的交叉研究(AI for Science),在Physics-Informed Learning及高维科学计算方面积累了丰富的经验。

1. **本构关系发现**:前期工作中,我们提出了一种基于PINNs的微观-宏观本构关系发现方法(Micro-to-Macro PINNs)。该方法能够直接从带噪声的应力-应变数据中,自动发现隐含的本构模型参数,相比传统有限元反演方法效率提升了100倍。相关成果已整理为论文 "Micro-to-Macro Physics-Informed Neural Constitutive Discovery"。

2. **生成式建模**:我们在生成对抗网络(GANs)与物理约束的结合方面进行了探索性研究,初步验证了在生成器Loss中加入物理正则项可以显著提升生成样本的物理真实度。这些工作为本项目提出的“流形约束生成器”奠定了坚实基础。

\subsubsection*{(2)已取得的研究工作成绩}
- **论文发表**:在相关领域发表SCI论文XX篇,包括Journal of Computational Physics, Nature Reviews Physics等。
- **专利申请**:已申请相关发明专利XX项。
- **代码开源**:在GitHub上开源了多个AI4S相关工具包,获得Star数超过XXX。
