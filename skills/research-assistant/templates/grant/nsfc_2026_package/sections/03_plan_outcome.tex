\subsection*{1.3 年度研究计划及预期研究成果}

\subsubsection*{(1)年度研究计划}
\textbf{第一年(2027年):理论构建与算法验证} \\
- 1-6月:构建高维物理流形的数学描述,推导生成模型的几何约束条件。
- 7-12月:开发“流形约束生成器”原型算法,在标准物理基准数据集(如Darren-Stokes流)上验证物理一致性。

\textbf{第二年(2028年):双螺旋架构实现与本构发现} \\
- 1-6月:实现“双螺旋”优化算法,开展微观-宏观本构关系的挖掘实验。
- 7-12月:引入符号回归模块,打通从神经网络到解析公式的可解释链路。

\textbf{第三年(2029年):模型泛化与平台集成} \\
- 1-6月:开展跨介质泛化实验(岩石->生物组织),优化通用基础模型性能。
- 7-12月:集成所有算法模块,开发可视化AI4S科研平台,整理代码与数据并开源。

\subsubsection*{(2)预期研究成果}
1. **理论成果**:建立一套完整的“流形约束物理生成理论(Manifest Theory)”,发表顶级期刊/会议论文(NeurIPS, ICML, Nature Comm等)5-8篇。
2. **技术成果**:申请国家发明专利 3-5 项,形成具有自主知识产权的核心算法库。
3. **系统成果**:发布开源“通用物理生成计算平台”一套,服务于社区。
4. **人才培养**:培养博士研究生 2-3 名,硕士研究生 3-5 名。
