\subsection*{1.2 本项目拟解决的关键科学问题、拟采取的研究方案及可行性分析}

\subsubsection*{(1)关键科学问题}
\textbf{问题一:如何在高维生成模型的隐空间中严格嵌入非线性物理守恒约束?} \\
现有PINNs方法多采用软约束(Penalty)形式,在高维空间中难以保证严格满足。如何利用微分几何流形理论,构建硬约束(Hard Constraint)的生成算子,是实现物理一致性的核心难点。

\textbf{问题二:在数据稀疏与噪声干扰下,如何保证“双螺旋”优化架构的收敛性与鲁棒性?} \\
在微观实验数据极其有限(Small Data)的情况下,如何平衡数据拟合项与物理正则项的梯度冲突,避免优化陷入局部极小值,并从理论上证明本构发现的唯一性。

\textbf{问题三:如何实现跨物理体制(Regime)的生成模型泛化?} \\
如何设计通用的网络算子,使其不仅适用于单一材料体系,还能通过少样本微调(Few-Shot Finetuning)快速适应全新的物理介质(如从岩石力学泛化到生物力学),突破传统模型的领域壁垒。

\subsubsection*{(2)拟采取的研究方案}
\textbf{方案一:流形约束生成器(Manifold-Constrained Generator)} \\
1. **构造物理流形算子**:利用辛几何网络(Symplectic Networks)对哈密顿系统进行建模,自然保持能量守恒。
2. **投影梯度下降(Projected Gradient Descent)**:在生成过程的每一步,强制将样本投影回物理允许的流形空间。

\textbf{方案二:双螺旋(Dual-Helix)迭代优化策略} \\
1. **双流交互**:构建Data-Stream(处理观测数据)与Physics-Stream(处理守恒方程)双通道。
2. **动态加权**:引入自适应权重机制(Uncertainty Weighting),根据两个流的互信息动态调整梯度贡献,确保训练稳定收敛。
3. **符号回归蒸馏**:在训练末期,利用符号回归(Symbolic Regression)从神经网络中提取解析表达式,实现究极的可解释性。

\textbf{方案三:基于神经算子(Neural Operator)的通用架构} \\
1. 引入Fourier Neural Operator (FNO) 作为骨干网络,学习解算子而非单一解,实现分辨率无关的泛化。
2. 构建预训练-微调(Pretrain-Finetune)范式,在大规模合成数据集上预训练物理先验,在小样本实验数据上微调。

\subsubsection*{(3)可行性分析}
1. **理论基础扎实**:申请人团队在PINNs及微分几何领域有深厚积累,前期已发表相关高水平论文[Raissi2019]。
2. **技术储备充分**:已初步实现了“Micro-to-Macro PINNs”原型系统,验证了本构发现的可行性(见附件预实验报告)。
3. **软硬件条件**:拥有完善的AI4S计算集群(NVIDIA A100)及开源代码库支持。
