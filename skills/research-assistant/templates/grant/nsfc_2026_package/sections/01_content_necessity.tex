\subsection*{1.1 本项目研究内容、项目开展的必要性及与重大研究计划总体科学目标的关系}

\subsubsection*{(1)研究开展的必要性}
AI for Science(AI4S)正在重塑科学发现范式,但当前主流的“黑盒”深度学习模型在解决高维复杂物理问题时面临**严峻挑战**:(1) **缺乏物理可解释性**:纯数据驱动模型难以保证热力学一致性(如能量守恒、熵增原理),导致预测结果在非采样区域(OOD)违反物理定律;(2) **泛化能力受限**:对于具有复杂几何约束(如微观结构流形)的材料设计问题,传统生成模型生成的样本往往偏离物理流形,无法应用于实际制造。

针对国家自然科学基金重大研究计划“可解释、可通用的下一代人工智能方法”中的核心方向——**“高维复杂物理约束下的生成建模与科学计算”**,本项目提出一种融合物理守恒律与几何流形先验的生成式人工智能新范式。通过建立**流形约束下的物理生成理论(Manifest)**,解决从微观本构发现到宏观材料设计的跨尺度一致性难题,对于推动我国在计算材料学与下一代AI基础理论的交叉融合具有重要的战略意义。

\subsubsection*{(2)研究内容}
本项目拟围绕“**面向复杂材料设计的流形约束物理生成式人工智能方法**”这一核心科学目标,开展以下三个方面的研究:

\textbf{内容一:基于流形几何的高维物理守恒律嵌入机制} \\
研究如何将热力学守恒定律(质量、动量、能量及熵不等式)转化为高维数据流形上的几何约束。提出“物理流形映射网络”,确保生成模型仅在满足物理定律的低维流形上进行采样,从根本上消除“物理幻觉”。

\textbf{内容二:可解释的双螺旋(Dual-Helix)本构发现优化架构} \\
针对微观-宏观跨尺度本构关系发现难题,设计数据驱动与物理驱动交替迭代的“双螺旋”优化算法。证明该算法在稀疏数据下的收敛性与唯一性,实现本构关系的显式解析表达,赋予神经网络模型完全的可解释性。

\textbf{内容三:面向通用的跨尺度材料生成基础模型} \\
构建融合结构先验的通用生成基础模型,实现从颗粒材料到生物软组织等不同介质的跨领域迁移。验证模型在少样本条件下的泛化能力,形成开源可复用的AI4S科学计算平台。

\subsubsection*{(3)与重大研究计划总体科学目标的关系}
本项目紧扣指南方向3:“高维复杂物理约束下的生成建模”。
1. **契合度**:直接响应指南关于“物理一致性缺失”与“流形约束”的痛点。
2. **贡献点**:提出的“双螺旋”架构与“物理流形嵌入”是对下一代可解释AI方法的理论创新。
