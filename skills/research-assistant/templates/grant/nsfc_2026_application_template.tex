\documentclass[a4paper,12pt]{article}
\usepackage{ctex}
\usepackage{geometry}
\usepackage{graphicx}
\usepackage{titlesec}
\usepackage{xcolor}
\usepackage{fancyhdr}

% Page Setup
\geometry{left=2.5cm, right=2.5cm, top=2.5cm, bottom=2.5cm}
\pagestyle{fancy}
\fancyhf{}
\cfoot{\thepage}

% NSFC Style Sectioning
\titleformat{\section}{\large\bfseries}{(\textbf{\chinese{section}})}{0.5em}{}
\titleformat{\subsection}{\normalsize\bfseries}{\arabic{subsection}.}{0.5em}{}

\title{报告正文(交叉专用 2026版)}
\author{申请人:[姓名]}
\date{\today}

\begin{document}
\maketitle

\noindent\textbf{\color{red}参照以下提纲撰写,要求内容翔实、主题突出。请勿删除或改动下述提纲标题及括号中的文字。}

\section{重点阐述内容(建议不超过5000字)}

\subsection{本项目的研究内容、项目开展的必要性及与重大研究计划总体科学目标的关系}
% [在此处撰写]
% 指南方向:可解释、可通用的下一代人工智能方法
% 关联:重点支持项目/集成项目

\subsection{本项目拟解决的关键科学问题、拟采取的研究方案(重点阐述研究方案的特色或创新之处)及可行性分析}
% [在此处撰写]
% 关键科学问题:
% 1. ...
% 2. ...
% 研究方案:
% ...

\subsection{年度研究计划及预期研究成果}
% [在此处撰写]
% 2027.01 - 2027.12: ...
% 2028.01 - 2028.12: ...
% 2029.01 - 2029.12: ...
% 预期成果:
% 1. 论文 ... 篇
% 2. 专利 ... 项
% 3. 系统/平台 ...

\section{研究基础与工作条件}

\subsection{研究基础(与本项目相关的研究工作积累和已取得的研究工作成绩)}
% [在此处撰写]

\subsection{正在承担的与本项目相关的科研项目情况}
% (申请人和主要参与者正在承担的与本项目相关的科研项目情况,包括国家自然科学基金的项目和国家其他科技计划项目,要注明项目的资助机构、项目类别、批准号、项目名称、获资助金额、起止年月、与本项目的关系及负责的内容等)

\subsection{完成国家自然科学基金项目情况}
% (对申请人负责的前一个已资助期满的科学基金项目(项目名称及批准号)完成情况、后续研究进展及与本申请项目的关系加以详细说明。另附该项目的研究工作总结摘要(限500字)和相关成果详细目录)

\section{其他需要说明的情况}

\subsection{申请人同年申请不同类型的国家自然科学基金项目情况}
% [如无,填写“无”。]

\subsection{具有高级专业技术职务(职称)的申请人或者主要参与者是否存在同年申请或者参与申请国家自然科学基金项目的单位不一致的情况}
% [如无,填写“无”。]

\subsection{具有高级专业技术职务(职称)的申请人或者主要参与者是否存在与正在承担的国家自然科学基金项目的单位不一致的情况}
% [如无,填写“无”。]

\subsection{申请人和主要参与者同年以不同专业技术职务(职称)申请或参与申请科学基金项目的情况}
% [如无,填写“无”。]

\subsection{其他}
% [如无,填写“无”。]

\end{document}
