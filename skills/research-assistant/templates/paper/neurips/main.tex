\documentclass{article}

% if you need to pass options to natbib, use, e.g.:
%     \PassOptionsToPackage{numbers, compress}{natbib}
% before loading neurips_202X

% ready for submission
\usepackage{neurips_2026}


% to compile a preprint version, e.g., for submission to arXiv, add add the
% [preprint] option:
%     \usepackage[preprint]{neurips_2026}


% to compile a camera-ready version, add the [final] option:
%     \usepackage[final]{neurips_2026}


% to avoid loading the natbib package, add option nonatbib:
%    \usepackage[nonatbib]{neurips_2026}


\usepackage[utf8]{inputenc} % allow utf-8 input
\usepackage[T1]{fontenc}    % use 8-bit T1 fonts
\usepackage{hyperref}       % hyperlinks
\usepackage{url}            % simple URL typesetting
\usepackage{booktabs}       % professional-quality tables
\usepackage{amsfonts}       % blackboard math symbols
\usepackage{nicefrac}       % compact symbols for 1/2, etc.
\usepackage{microtype}      % microtypography
\usepackage{xcolor}         % colors


\title{Formatting Instructions For NeurIPS 2026}


% The \author macro works with any number of authors. There are two commands
% used to separate the names and addresses of multiple authors: \And and \AND.
%
% Using \And between authors leaves it to LaTeX to determine where to break the
% lines. Using \AND forces a line break at that point. So, if LaTeX puts 3 of 4
% authors names on the first line, and the last on the second line, try using
% \AND instead of \And before the third author name.


\author{%
  David S.~Hippocampus\thanks{Use footnote for providing further information
    about author (webpage, alternative address)---\emph{not} for acknowledging
    funding agencies.} \\
  Department of Computer Science\\
  Cranberry-Lemon University\\
  Pittsburgh, PA 15213 \\
  \texttt{hippo@cs.cranberry-lemon.edu} \\
  % examples of more authors
  % \And
  % Coauthor \\
  % Affiliation \\
  % Address \\
  % \texttt{email} \\
  % \AND
  % Coauthor \\
  % Affiliation \\
  % Address \\
  % \texttt{email} \\
  % \And
  % Coauthor \\
  % Affiliation \\
  % Address \\
  % \texttt{email} \\
  % \And
  % Coauthor \\
  % Affiliation \\
  % Address \\
  % \texttt{email} \\
}


\begin{document}


\maketitle


\begin{abstract}
  The abstract paragraph should be indented \nicefrac{1}{2}~inch (3~picas) on
  both the left- and right-hand margins. Use 10~point type, with a vertical
  spacing (leading) of 11~points.  The word \textbf{Abstract} must be centered,
  bold, and in point size 12. Two line spaces precede the abstract. The abstract
  must be limited to one paragraph.
\end{abstract}


\section{Submission of papers to NeurIPS 2026}


Please read the instructions below carefully and follow them faithfully.


\section{Style}


Papers to be submitted to NeurIPS 2026 must be prepared according to the
instructions presented here.


%% Please note that we have introduced automatic line number generation
%% into the style file for \LaTeXe. This is to help reviewers
%% refer to specific lines of the paper when they make their comments. Please do
%% NOT refer to these line numbers in your paper as they will be removed from the
%% style file for the final version of accepted papers.


Authors are required to use the NeurIPS \LaTeX{} style files obtainable at the
NeurIPS website. Please make sure you use the current files and
not previous versions. Tweaking the style files may be grounds for rejection.


\section{Retrieval of style files}


The style files for NeurIPS and other conference information are available online at:
\begin{center}
   \url{http://www.neurips.cc/}
\end{center}
The file \verb+neurips_2026.pdf+ contains these
instructions and illustrates the
various formatting requirements your NeurIPS paper must satisfy.


The only supported style file for NeurIPS 2026 is \verb+neurips_2026.sty+,
rewritten for \LaTeXe{}.  \textbf{Previous style files for \LaTeX{} 2.09,
  Microsoft Word, and RTF are not supported!}


The \LaTeX{} style file contains three optional arguments: \verb+final+,
which creates a camera-ready copy, \verb+preprint+, which creates a
preprint for submission to, e.g., arXiv, and \verb+nonatbib+, which will
not load the \verb+natbib+ package. The \verb+preprint+ option behaves
like the \verb+final+ option, except that it adds a ``Preprint. Work in
progress.'' text in the footer. Do not use the \verb+final+ option for
submissions.


\begin{ack}
Use unnumbered first level headings for the acknowledgments. All acknowledgments
go at the end of the paper before the list of references. Moreover, you are required to declare
funding (financial activities supporting the submitted work) and competing interests (related financial activities outside the submitted work).
More information about this disclosure can be found at: \url{https://neurips.cc/Conferences/2026/PaperInformation/FundingDisclosure}.


Do {\bf not} include this section in the anonymized submission, only in the final paper. You can use the \texttt{ack} environment provided in the style file to autmoatically hide this section in the anonymized submission.
\end{ack}

% References
\bibliographystyle{plain}
\bibliography{refs}


\end{document}
